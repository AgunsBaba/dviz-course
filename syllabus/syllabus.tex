\documentclass[11pt,article,oneside]{memoir} %{{{
% based on Kieran Healy's syllabus templates
% https://github.com/kjhealy/latex-custom-kjh 

\usepackage{org-preamble-pdflatex} 

\setlength{\parskip}{10pt}
\setlength{\parindent}{0pt}

%}}}
% Definitions %{{{
\def\myauthor{Author}
\def\mytitle{Title}
\def\mycopyright{\myauthor}
\def\mykeywords{}
\def\mybibliostyle{plain}
\def\mybibliocommand{}
\def\mysubtitle{}
\def\myaffiliation{Indiana University}
\def\myaddress{Info East Rm 316} 
\def\myemail{yyahn@iu.edu}
\def\myweb{http://yongyeol.com}
\def\myphone{856-2920}
\def\myversion{}
\def\myrevision{}

\def\myaffiliation{\ \\Indiana University}
\def\myauthor{Yong-Yeol (YY) Ahn}
\def\mykeywords{Visualization, Data, Undergraduate, Informatics}
\def\mysubtitle{Syllabus}
\def\mytitle{{\normalsize \textsc{Info} 422/I590 \newline} \HUGE Data Visualization}

%%\chapterstyle{article-3}
%\pagestyle{kjh}

\def\ind{\hangindent=1 true cm\hangafter=1 \noindent}
\def\labelitemi{$\cdot$}

\chapterstyle{article-4}  % alternative styles are defined in latex-custom-kjh/needs-memoir/

%}}}
\begin{document} %{{{

\title{\LARGE \mytitle} %{{{
\author{\Large\myauthor \newline \footnotesize\texttt{\noindent\myemail}}
%\date{Fall 2017. Info West 107 (M) / 109 (W)\newline MW 4:00pm--5:15pm. \newline Office hours: W 9am-10am}

%\published{\sffamily I590/H400/I400 / Fall 2014 / Mon \& Wed 4:00--5:15pm / Info West 107 (M) \& 109 (W)}
\maketitle 

\vspace{-20pt}{\bfseries Assistant Instructors} \\ Silvia Karim (\texttt{sikarim@iu.edu})  \\ &
Abu Saleh Md Noman (\texttt{amdnoman@indiana.edu}) 
%}}}
\section{Communication} %{{{

Announcements and other communication will be through Slack: 

\url{https://iu-dviz-course.slack.com}

Therefore, you should join as soon as possible. You can join with your IU email adddress by visiting the following link. 

\url{https://join.slack.com/t/iu-dviz-course/signup}

%}}}
\section{Course Description}%{{{

From TV news to cutting-edge scientific papers, from a home office to the
largest companies in the world, data visualization is extensively used to
reveal patterns in data and to tell stories. More and more data is collected
through devices and services, and more and more decisions are made through
data. Data visualization is indispensible for data analysis, and thus is an
essential skill for every knowledge worker.  This course is an introduction to
basic statistical data analysis and visualization.  We will learn fundamentals
of data visualization in the context of perception, integrity, design,
statistics, types of data, and visualization techniques.  The hands-on
exercises using the Python stack will be an integral part of the course. 

\paragraph{Relationships with S637 Information Visualization (IVMOOC):}
Compared with S637, this course is geared more towards fundamental statistical
visualizations and exploratory data analysis, using the Python data science and
visualization stack.  Therefore, this course may be more suitable for students who
pursue their careers in research, developement, and data analysis. 

%}}}
\section{Course Objectives}%{{{

By the end of the course, you are expected to be able to understand, explain,
and manipulate basic types of data, analyze them by applying basic exploratory
visualization techniques, and create basic explanatory web-based
visualizations. You will also be able to evaluate the effectiveness of data
visualizations based on the principles of human perception, design, types of
data, and visualization techniques. 
 
%\begin{itemize} \item Exploratory data visualization and analysis

%\begin{itemize} \item Python data analysis stack \item Data types \item
%Loading and manipulating data \item Fundamental data visualizations
%\end{itemize}

%\end{itemize} }}}
\section{Prerequisites}%{{{
\label{sec:Prerequisites}

This course's residential version is open to advanced undergraduate students as
well as graduate students; online version of this course is open to graduate
students. Because creating visualizations using programming languages (Python)
is an integral part of the course, it is required to have good understanding
and working knowledge of programming. 

For the undergraduate students, the prerequisites are:
%
\begin{itemize}\vspace{-10pt}
%
    \item Programming foundations (I210 \& I211, or equivalent courses). 
%
\end{itemize}\vspace{-10pt}
%
It is strongly recommended to have basic understanding of mathematics,
statistics, and Web. The following courses are recommended before taking this
course: 
%
\begin{itemize}\vspace{-10pt}
%
\item I123: Data Fluency \item I300: HCI/Interaction Design \item I308:
Information Representation \item I360: Web Design
%
\end{itemize}\vspace{-10pt}

For self-assessment, visit the following link: \href{http://bit.ly/dviz\_self}{http://bit.ly/dviz\_self}. 
Contact the instructor if you are uncertain about your background. 
%}}}
\section{Expectations and Requirements}%{{{
\label{sec:requirements}

\paragraph{(All sections)} The final assessment will be mainly based on the exam and course project. The choice of
project topic can be guided by the instructors but you will have freedom to
choose projects topics. You are required to submit a final paper that contains
detailed explanation of the visualization process and results as well as the
visualization artifact itself (e.g. a visualization tool or a webpage)
depending on the nature of your project. 

\paragraph{(Residential course)} You are expected to attend every class and engage in discussions. You are not
allowed to use your phone or computer during the class unless explicitly asked to do so.
You are expected to read reading materials \emph{prior} to the class. At the
beginning of most class meetings, there will be an \emph{in-class quiz} based
on the assigned readings and previous materials. You
are expected to complete all weekly assignments. 

\paragraph{(Online)} You are expected to complete all course modules and assignments. You are also expected to 
actively engage in discussions on Slack. 

%}}}
\section{Books and key materials}%{{{

There is no required textbook, but the following books and websites are
recommended.

\subsection{Python and data analysis}%{{{

\begin{enumerate}%{{{

\item \href{https://docs.python.org/3/}{Python 3 Official Documentation}

\item \href{http://www.diveintopython3.net/index.html}{Dive Into Python} by Mark Pilgrim (available online) 

\item \href{http://work.thaslwanter.at/Stats/html/}{An introduction to statistics} (with Python) by Thomas Haslwanter (available online): this book uses Python to explain basic statistics. It also contains a succinct tutorial for Python and data visualization using Python. 

\item \href{http://www.learnpython.org}{Learnpython.org}: A web-based interactive tutorial 

\item \href{http://ipython.rossant.net}{Learning IPython for Interactive Computing and Data Visualization} by  Cyrille Rossant: Introduction to IPython as well as lots of advanced analysis 


\end{enumerate}%}}}
%}}}
\subsection{Visualization and Design}%{{{

See \href{http://yyahnwiki.appspot.com/Information_visualization#h_6225eb5bf8a031f750a1b03f810ccc6a}{Visualization books} on my wiki. 

%\begin{enumerate}%{{{

%\item \href{http://www.amazon.com/gp/product/0961392142}{The Visual Display of Quantitative Information (2nd ed.)} by E.R. Tufte: one of the foundational book on visualization. It contains a rich set of historical visualization, thoughtful discussion on visualization principles. 

%\item \href{http://www.amazon.com/Atlas-Knowledge-Anyone-Can-Map/dp/0262028816}{Atlas of Knowledge: Anyone Can Map} by K. Börner: this book systematically analyzes vocabularies of visualization with a lot of great examples. 

%\item \href{http://www.amazon.com/Visualization-Analysis-Design-AK-Peters/dp/1466508914}{Visualization Analysis and Design} by T. Munzner: a nice textbook that covers important topics of visualization. 

%\item \href{http://www.amazon.com/Visual-Thinking-Kaufmann-Interactive-Technologies/dp/0123708966}{Visual Thinking for Design} by C. Ware: one of the best books on the role of perception in visualization. 

%\end{enumerate}%}}}
%\subsection{D3.js}%{{{

%\begin{enumerate}%{{{

%\item \href{http://www.amazon.com/Interactive-Data-Visualization-Scott-Murray/dp/1449339735}{Interactive Data Visualization for the Web} by Scott Murray

%\item \href{https://github.com/d3/d3/wiki/Gallery}{D3 Gallery}, \href{https://github.com/d3/d3/wiki/tutorials}{D3 Tutorials}, and \href{https://bl.ocks.org/mbostock}{mbostock's Blocks}. 

%\end{enumerate}%}}}
%}}}
%}}}

%}}}
\section{Policies}%{{{

\begin{enumerate}%{{{

\item \emph{Disabilities.} Every attempt will be made to accommodate qualified
students with disabilities (e.g. mental health, learning, chronic health,
physical, hearing, vision, neurological, etc.). You must have established your
eligibility for support services through Disability Services for Students. Note
that services are confidential, may take time to put into place, and are not
retroactive.  Captions and alternate media for print materials may take three
or more weeks to get produced. Please contact Disability Services for Students
at \url{http://disabilityservices.indiana.edu} or 812-855-7578 as soon as
possible if accommodations are needed. The office is located on the third
floor, west tower, of the Wells Library (Room W302). Walk-ins are welcome 8 AM
to 5 PM, Monday through Friday. You can also locate a variety of campus
resources for students and visitors who need assistance at
\url{http://www.iu.edu/~ada/index.shtml}. 

\item \emph{Sexual misconduct and Title IX.} As your instructor, one of my
responsibilities is to create a positive learning environment for all students.
Title IX and IU's Sexual Misconduct Policy prohibit sexual misconduct in any
form, including sexual harassment, sexual assault, stalking, and dating and
domestic violence.  If you have experienced sexual misconduct, or know someone
who has, the University can help. If you are seeking help and would like to
speak to someone confidentially, you can make an appointment with:

\begin{enumerate}
    
\item The Sexual Assault Crisis Services (SACS) at (812) 855-8900 (counseling services)
\item Confidential Victim Advocates (CVA) at (812) 856-2469 (advocacy and advice services)
\item IU Health Center at (812) 855-4011 (health and medical services)

\end{enumerate}

It is also important that you know that Title IX and University policy require
me to share any information brought to my attention about potential sexual
misconduct, with the campus Deputy Title IX Coordinator or IU's Title IX
Coordinator. In that event, those individuals will work to ensure that
appropriate measures are taken and resources are made available. Protecting
student privacy is of utmost concern, and information will only be shared with
those that need to know to ensure the University can respond and assist. I
encourage you to visit \emph{stopsexualviolence.iu.edu} to learn more. 


\item \emph{No electronics---laptops, tablets, and smartphones---may be used in
class}, unless the usage is specifically requested by the instructors.  It has
been shown that
\href{http://www.scientificamerican.com/article/a-learning-secret-don-t-take-notes-with-a-laptop/}{using
laptops in class hurts learning, \emph{even if} you are using it to take
notes}.  If you must have electronics due to a special reason, please obtain
permission beforehand. 

\item \emph{Be honest.} Your assignments and papers should be your own work.
If you find useful resources for your assignments, share them and cite them. If
your friends helped you, acknolwedge them. You should feel free to discuss both
online and offline, but do not show your code directly.  Any cases of academic
misconduct (cheating, fabrication, plagiarism, etc) will be reported to the
School and the Dean of Students, following the standard procedure. Cheating is
not cool. 

\item \emph{You have the responsibility of backing up all your data and code}.
Always use at least Box, Dropbox, or Google Drive. Ideally, learn version
control systems and use \url{https://github.iu.edu} or
\url{https://github.com}. Loss of data, code, or papers due to various reasons
(e.g. malfunction of your laptop) is not an acceptable excuse for delayed or
missing submission. 

\item \emph{Inform your excused absences prior to class}. Please contact the
instructor until the previous day for an excused absence.  

\item If you have any issues, don't hesistate to contact me or
\href{http://healthcenter.indiana.edu/counseling/index.shtml}{IU's Counseling
and Psychological Services}. 


\end{enumerate}%}}}
%}}}
\section{Grading}%{{{
\label{sec:grading_tentative_}

\begin{itemize}%{{{

\item Attendance, Quiz, and Participation: 30\%

\item Assignments: 30\%

\item Final project: 40\%

\end{itemize}%}}}
%}}}
\section{Course Schedule}%{{{

The schedule is subject to change. 

%Find the most up-to-date schedule as well as
%detailed notes at \url{https://github.com/yy/dviz-course/wiki/Schedule}.

\subsection{Week 1 (5/7-): Why visualization? }
\subsection{Week 2 (5/14-): History and integrity  }
\subsection{Week 3 (5/21-): Perception }
\subsection{Week 4 (5/28-): Design }
\subsection{Week 5 (6/4-): Data Types and 1-D data }
\subsection{Week 6 (6/11-): Histogram and Boxplot } 
\subsection{Week 7 (6/18-): Estimation }
\subsection{Week 8 (6/25-): Logscale and Beyond 1-D }
\subsection{Week 9 (7/2-): High-dimensional data }
\subsection{Week 10 (7/9-): Maps }
\subsection{Week 11 (7/16-): Text and Networks}
\subsection{Week 12 (7/23-): Project hack \& presentation }
%\subsection{Week 13 Texts and Graphs}
%\subsection{Week 14 Thanksgiving}
%\subsection{Week 15 Graphs II }
%\subsection{Week 16 Final project presentation}
%}}}

\end{document} %}}}
